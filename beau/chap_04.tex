\chapter{数值解}
对于大部分的偏微分方程模型问题来说,解得它们的解析解是比较困难的.在大部分的情况下,我们不能也没有必要求得它们的解析解.
因此,研究它们的数值解的求解方法是很有必要的.\par
在这一章中,我们同样从简单的方程开始解,建立数值解的求解方法的基本体系,然后我们再研究复杂模型的求解方法.
\section{常系数扩散方程}
考虑常系数扩散方程
\begin{equation}\label{eq:04_cxsks_o}
\begin{cases}
 \dfrac{\partial u}{\partial t}=a\dfrac{\partial^2 u}{\partial x^2} \\
 u(x,0)=g(x),\quad x\in\mathbb{R}
\end{cases}
\end{equation}
构成了初值问题,其向前差分格式\upcite{王新艳2001}为
\begin{gather}
 \dfrac{u_{j}^{n+1}-u_j^n}{\tau}-a\dfrac{u_{j+1}^n-2u_j^n+u_{j-1}^n}{h^2}=0,\label{eq:04_cxsks_xq}\\
 u_j^0 = g(x_j)
\end{gather}
其截断误差为$O(\tau+h^2)$.\par
考虑它的向后差分格式
\begin{equation}\label{eq:04_cxsks_xh}
 \dfrac{u_j^n-u_j^{n-1}}{\tau}-a\dfrac{u_{j+1}^n-2u_j^n+u_{j+1}^n}{h^2}=0
\end{equation}
其截断误差也是$O(\tau+h^2)$.
\subsection{加权隐式格式}
将式~\eqref{eq:04_cxsks_xq}~改写为
\begin{equation}\label{eq:04_cxsks_xq_gx}
 \dfrac{u_j^n-u_j^{n-1}}{\tau}-a\dfrac{u_{j+1}^{n-1}-2u_{j}^{n-1}+u_{j-1}^{n-1}}{h^2}=0
\end{equation}
在式~\eqref{eq:04_cxsks_xq}~乘以$\theta$,用$(1-\theta)$乘以~\eqref{eq:04_cxsks_xq_gx},得到差分格式
\begin{equation}\label{eq:04_cxsks_js}
 \dfrac{u_j^n-u_j^{n-1}}{\tau}-a\left[\theta\dfrac{u_{j+1}^n-2u_j^n+u_{j-1}^n}{h^2}
 +(1-\theta)\dfrac{u_{j+1}^{n-1}-2u_j^{n-1}+u_{j-1}^{n-1}}{h^2}\right]=0
\end{equation}
其中,~$0\leq\theta\leq 1$,这种差分格式称为加权差分格式\upcite{王玉珉2004},将其改写为易于计算的格式
\begin{multline}
 -a\lambda\theta_{j+1}^n+(1+2a\lambda\theta)u_j^n-a\lambda\theta u_{j-1}^n = \\
 a\lambda(1-\theta)u_{j+1}^{n-1}+[1-2a\lambda(1-\theta)]u_j^{n-1}+a\lambda(1-\theta)u_{j-1}^{n-1}
\end{multline}
其中,~$\lambda=\tau/h^2$,我们求差分格式~\eqref{eq:04_cxsks_js}~的截断误差,设$u(x,t)$是方程~\eqref{eq:04_cxsks_o}~的
充分光滑的解,在$(x_j,t_n)$处进行Taylor级数展开得
\begin{equation*}
E=a\left(\dfrac{1}{2}-\theta\right)\tau\left[\dfrac{\partial^3 u}{\partial x^2}{\partial t}\right]_j^n
+O(\tau^2+h^2)
\end{equation*}
当$\theta\not=1/2$时,其截断误差为$O(\tau+h^2)$.当$\theta=1/2$时,其截断误差为$O(\tau^2+h^2)$.\par
采用Fourier方法分析\upcite{刘乃伟2009}差分格式~\eqref{eq:04_cxsks_js}~的稳定性,求得其增长因子为
\begin{equation}
 G(\tau,k)=\dfrac{1-4(1-\theta)a\lambda\sin^2\dfrac{kh}{2}}{1+4\theta a\lambda\sin^2\dfrac{kh}{2}}
\end{equation}
我们给出用于判断差分格式稳定性的\textbf{von Neumann}~定理,即
\begin{Theorem}[von Neumann定理]
差分格式
\begin{equation}
	u^{n+1}_j=C(x_j,\tau)u^n_j
\end{equation}
稳定的必要条件是当$\tau\leq\tau_0$,$n\tau\leq T$,对所有的k有
\begin{equation}
	\left|\lambda_j(G(\tau,k))\right| \leq 1+M\tau,\quad j=1,2,\cdots,p,
\end{equation}
其中$\left|\lambda_j(G(\tau,k))\right|$表示$G(\tau,k)$的特征值,$M$为常数。
\end{Theorem}\par
因此,要$|G(\tau,k)|\leq 1$,即
\begin{equation*}
 -1\leq\dfrac{1-4(1-\theta)a\lambda\sin^2\dfrac{kh}{2}}{1+4\theta a\lambda\sin^2\dfrac{kh}{2}}\leq 1
\end{equation*}
考虑左边的不等式,得
\begin{equation*}
4a\lambda(1-2\theta)\sin^2\dfrac{kh}{2}\leq2
\end{equation*}
因为$\sin^2\dfrac{kh}{2}\leq 1$,要求化为
\begin{equation}
 2a\lambda(1-2\theta)\leq 1.
\end{equation}
这是~\eqref{eq:04_cxsks_js}~的稳定性要求.
\subsection{三层隐式格式}
考虑三层隐式差分格式\upcite{张培文2002},考虑
\begin{equation}\label{eq:04_cxsks_scys}
\dfrac{3}{2}\dfrac{u_j^{n+1}-u_j^n}{\tau}-\dfrac{1}{2}\dfrac{u_j^n-u_j^{n-1}}{\tau}
-a\dfrac{u_{j+1}^{n+1}-2u_{j}^{n+1}+u_{j-1}^{n+1}}{h^2}=0
\end{equation}
改写成便于计算的格式
\begin{equation}
 (3+4a\lambda)u_j^{n+1}-2a\lambda(u_{j+1}^{n+1}+u_{j-1}^{n+1})=4u_j^n-4u_j^{j-1}
\end{equation}
其中,~$\lambda=\tau/h^2$,易证差分格式~\eqref{eq:04_cxsks_scys}~以二阶的精度逼近原微分方程.\par
将其化为等价的二层差分方程组
\begin{equation*}
\begin{cases}
(3+4a\lambda)u_j^{n+1}-2a\lambda(u_{j+1}^{n+1}+u_{j-1}^{n+1})=4u_j^n-v_j^n \\
v_j^{n+1}=u_j^n
\end{cases}
\end{equation*}
求得其增长矩阵为
\begin{equation*}
 G(\tau,k)=\begin{bmatrix}
  \dfrac{4}{3+8a\sin^2\dfrac{kh}{2}} & \dfrac{-1}{3+8a\sin^2\dfrac{kh}{2}} \\
                     1		     &			0		   \\
 \end{bmatrix}
\end{equation*}
其中,~$a=a\lambda$,$G$的特征方程为
\begin{equation*}
 \mu^2-\dfrac{4}{3+8\alpha\sin^2\dfrac{kh}{2}}\mu+\dfrac{1}{3+8\alpha\sin^2\dfrac{kh}{2}}=0
\end{equation*}
得$|\mu_i|\leq1(i=1,2)$.写出$\mu_i$的表达式
\begin{equation*}
\mu_{1,2}=\dfrac{2\pm\sqrt{1-8\alpha\sin^2\dfrac{kh}{2}}}{3+8\alpha\sin^2\dfrac{kh}{2}}.
\end{equation*}
若解为重根,显然有$|\mu_i|$,由此得出差分格式~\eqref{eq:04_cxsks_scys}~是无条件稳定的.
\subsection{初边值问题}
我们考虑第一类边值问题,有
\begin{equation}
\begin{cases}
\dfrac{\partial u}{\partial t}-a\dfrac{\partial^2 u}{\partial x^2}=0,\quad 0<x<1,t>0 \\
u(x,0)=g(x),\quad 0\leq x\leq 1 \\
u(0,t)=\phi(t),\quad t\geq 0 \\
u(1,t)=\psi(t),\quad t\geq 0 
\end{cases}
\end{equation}
其计算区域为$x=[0,1]$,因此我们分开区间$0=x_0<x_1<\cdots<x_J=1$,第一边值问题的边界处理可取
\begin{equation}
\begin{cases}
u_0^n=\phi(t_n),\quad n\geq0 \\
u_J^n=\psi(t_n),\quad n\geq0
\end{cases}
\end{equation}\par
在初始线我们利用初始条件的离散
\begin{equation}
 u_j^0=g(x_j)=g_j
\end{equation}
得到边界点上的差分格式.\par
由于本研究不涉及第三类边值条件,这一方面的差分处理方法就不叙述了.
\section{常系数对流方程}
考虑简单的双曲型对流方程
\begin{equation}\label{eq:04_dl_o}
 \dfrac{\partial u}{\partial t}+a\dfrac{\partial u}{\partial x}=0
\end{equation}
我们来建立双曲型偏微分方程的求解差分格式.\par
首先讨论方程~\eqref{eq:04_dl_o}~的特征线,考虑$u$在直线$x-at=c$上的方向的方向导数,有
\begin{equation}
\left.\dfrac{\dif u}{\dif t}\right|_l = \dfrac{\partial u}{\partial t}+\dfrac{\partial u}{\partial t}
\dfrac{\partial x}{\partial t} = \dfrac{\partial u}{\partial t}+a\dfrac{\partial u}{\partial x}
\end{equation}
其中,由方程~\eqref{eq:04_dl_o}~知$u$沿$l$的值不变,这条直线即为方程的\emph{特征线}.\par
\subsection{差分格式的推导}
我们利用特征线方法\upcite{韦春霞2003}来构造差分格式.
\begin{figure}
\centering
\begin{tikzpicture}[scale=1.3,very thick]
\begin{scope}[thin]
\draw[-] (-0.5,0) -- (6.5,0) node[right]{第$n$层};
\draw[-] (-0.5,1.5) -- (6.5,1.5) node[right]{第$n+1$层};
 \foreach \l in {0,1.2,2.4,3.6,4.8,6.0} 
 {
   \draw (\l,-0.3) -- (\l,1.8);
 }
 \node[below] at (1.2,-0.3) {$m-2$};
 \node[below] at (2.4,-0.3) {$m-1$};
 \node[below] at (3.6,-0.35) {$m$};
 \node[below] at (4.8,-0.3) {$m+1$};
 \node[below] at (6.0,-0.3) {$m+2$};
 \node[above left] at (1.2,0) {$A$};
 \node[above left] at (2.4,0) {$B$};
 \node[above left] at (3.6,0) {$C$};
 \node[above left] at (4.8,0) {$D$};
\end{scope}
\draw[-] (3.0,0) node[below]{$Q$} -- (3.6,1.5) node[above left]{$P$};
\node[above] at (3.6,1.8) {$m$,$n+1$};
\end{tikzpicture}
\caption{用特征线法构造差分格式\label{fig:tzx}}
\end{figure}
如图~\ref{fig:tzx}~所示,我们假定第$n$层的$u_m^n$值已知,现在求$P$点$(m,n+1)$的值$u_m^{n+1}$.过$P$作特征线
与$n$时间层相交在$Q$点.我们假定$Q$在线段$BC$上,即$u(B)=u(C)$,采用下面的方法求$u(Q)$,即
\begin{asparaenum}
\item $u(B)$和$u(C)$线性插值,即
\begin{equation}
u(P)=u(Q)\approx u(B)\dfrac{\overline{CQ}}{h}+u(C)\dfrac{h-\overline{CQ}}{h}=u_{m-1}^n\dfrac{
a\Delta t}{h}+u_m^n\dfrac{h-a\Delta t}{h}
\end{equation}
即
\begin{equation}\label{eq:04_dl_yf}
 u_m^{n+1}=(1-ar)u_m^n+aru_{m-1}^n
\end{equation}
其中,~$r=\Delta t/h$,此格式称为\emph{迎风格式}\upcite{范世平2013},其截断误差为$O(\Delta t+h)$.
\item $u(B)$和$u(D)$两点线性插值,即
\begin{equation}
\begin{split}
u(P)&=u(Q)\approx\dfrac{1}{2}(1-ar)u(D)+\dfrac{1}{2}(1+ar)u(B) \\
    &=\dfrac{1}{2}(1-ar)u_{m+1}^n+\dfrac{1}{2}(1+ar)u_{m-1}^n
\end{split}
\end{equation}
即
\begin{equation}\label{eq:04_dl_laxf}
\begin{split}
u_{m+1}^n =& \dfrac{1}{2}(1-ar)u_{m+1}^n+\dfrac{1}{2}(1+ar)u_{m-1}^n \\
          =& \dfrac{1}{2}(u_{m-1}^n+u_{m+1}^n)-\dfrac{ar}{2}(u_{m+1}^n-u_{m-1}^n).
\end{split}
\end{equation}
称为Lax-Friedrichs格式.
\item $u(B)$,~$u(C)$和$u(D)$三点抛物线插值,即
\begin{equation}
\begin{split}
u(P)&=u(Q)\approx u(C)-ar[u(C)-u(B)]-\dfrac{ar(1-ar)}{2}[u(B)-2u(C)+u(D)] \\
    &=u(C)-\dfrac{ar}{2}(u(B)-u(D))+\dfrac{a^2 r^2}{2}[u(B)-2u(C)+u(D)] \\
    &=u_m^n-\dfrac{ar}{2}(u_{m+1}^n-u_{m-1}^n)+\dfrac{a^2 r^2}{2}(u_{m+1}^n
    -2u_m^n+u_{m-1}^n)
\end{split}
\end{equation}
即
\begin{equation}\label{eq:04_dl_laxw}
u_m^{n+1}=u_m^n-\dfrac{ar}{2}(u_{m+1}^n-u_{m-1}^n)+\dfrac{a^2 r^2}{2}
(u_{m+1}^n-2u_m^n+u_{m+1}^n)
\end{equation}
为Lax-Wendorff格式.
\end{asparaenum}\par
\subsection{精确度与稳定性分析}
对于式~\eqref{eq:04_dl_yf},相应地,容易建立下面的差分格式,
\begin{asparaitem}
\item 左偏心格式
\begin{equation}\label{eq:04_dl_lc}
 u_m^{n+1}=(1-ar)u_m^n+aru_{m-1}^n
\end{equation}
\item 右偏心格式 
  \begin{equation}\label{eq:04_dl_rc}
  u_{m}^{n+1}=(1+ar)u_m^n-aru_{m+1}^n        
  \end{equation}
\item 中心差分格式
  \begin{equation}\label{eq:04_dl_cc}
   u_{m+1}^n=u_m^n-\dfrac{1}{2}ar(u_{m+1}^n-u_{m-1}^n)
  \end{equation}
\end{asparaitem}
其中,左差分格式和右差分格式的截断误差为$O(\Delta t+h)$,中心差分格式为$O(\Delta t+h^2)$.\par
我们利用von Neumann稳定性分析方法来分析它们的稳定性.令$u_m^n=v^n\me^{\mi\sigma x}=v^n\me^{\mi\sigma mh}$,分别
代入式~\eqref{eq:04_dl_lc}---\eqref{eq:04_dl_cc}~中,增长因子有
\begin{asparaitem}
\item 左偏心格式
\begin{equation}
 G_1(\Delta t,\sigma)=ar\me^{-\mi\sigma h}+(1-ar)
\end{equation}
\item 右偏心格式
\begin{equation}
 G_2(\Delta t,\sigma)=(1+ar)-ar\me^{-\mi\sigma h}
\end{equation}
\item 中心差分格式
\begin{equation}
 G_3(\Delta t,\sigma)=1-\mi ar\sin\sigma h
\end{equation}
\end{asparaitem}\par
由此,我们得到这样的结论
\begin{asparaitem}
\item 左偏心格式的稳定性条件为$a>0$,且$a\Delta t/h\leq 1$.
\item 右偏心格式的稳定性条件为$a<0$,且$|a|\Delta t/h\leq 1$.
\item 中心差分格式在任何情况下都是不稳定的.
\end{asparaitem}\par
对于Lax-Friedrichs格式,其截断误差为
\begin{equation}
\dfrac{\Delta t}{h^2}\dfrac{\partial^2u_m^n}{\partial t^2}-\dfrac{h^2}{2\Delta t}\dfrac{\partial^2
u_m^n}{\partial^2 x}+O(h^2+\dfrac{\partial^4}{\Delta t})=O(\Delta t+\dfrac{h^2}{\Delta t}+h^2)
\end{equation}
在计算中,通常取$h/\Delta t$为常数,所以Lax-Friedrichs格式的精度为$O(\Delta t+h)$.令$u_m^n=v^n\me^{\mi\sigma x}=v^n\me^{\mi\sigma mh}$
代入~\eqref{eq:04_dl_laxf},得增长因子为
\begin{equation}
G(\Delta t,\sigma)=\dfrac{1}{2}(\me^{\mi\sigma h}+\me^{-\mi\sigma h})-\dfrac{ar}{2}(\me^{\mi\sigma h}+\me^{-\mi\sigma h})
=\cos\sigma h -\mi ar\sin\sigma h
\end{equation}
其中,~$r=\Delta t/h$,从而有
\begin{equation}
|G(\Delta t,\sigma)|^2=\cos^2 \sigma h + a^2r^2\sin^2\sigma h=1-(1-a^2 r^2)\sin^2\sigma h
\end{equation}
得稳定性条件为$|a|r\leq1$.
再考虑Lax-Wendroff格式,在$(m,n)$点处的截断误差为
\begin{equation}
\begin{split}
E_m^n &=\dfrac{\Delta t^2}{6}\left(\dfrac{\partial^3 u}{\partial t^3}\right)_m^n+
	\dfrac{ah^2}{6}\left(\dfrac{\partial^3 u}{\partial x^3}\right)_m^n-
	\dfrac{a^2\Delta t h^2}{24}+O(h^4+\Delta t^3) \\
      &=\dfrac{\Delta t^2}{6}\left(\dfrac{\partial^3 u}{\partial t^3}\right)_m^n+
	\dfrac{ah^2}{6}\left(\dfrac{\partial^3 u}{\partial x^3}\right)_m^n+
	O(\Delta th^2+\Delta t^3+h^4) \\
      &=\dfrac{\Delta t^2}{6}\left(\dfrac{\partial^3 u}{\partial t^3}\right)_m^n+
	\dfrac{ah^2}{6}\left(\dfrac{\partial^3 u}{\partial x^3}\right)_m^n+
	O(rh^3+\Delta t^3+h^4) \\
      &=O(\Delta t^2+h^2)
\end{split}
\end{equation}
我们看到,该格式是一个时间和空间二阶精度的格式.令$u_m^n=v^n\me^{\mi\sigma x}=v^n\me^{\mi\sigma mh}$,代入
式~\eqref{eq:04_dl_laxw}~得
\begin{equation}
\begin{split}
G(\Delta t,\sigma) &= 1-\dfrac{ar}{2}(\me^{\mi\sigma h}-\me^{-\mi\sigma h})+\dfrac{a^2r^2}{2}
(\me^{\mi\sigma h}-2+\me^{-\mi\sigma h}) \\
&= 1-2a^2 r^2 \sin^2\dfrac{\sigma h}{2}-\mi ar\sin\sigma h
\end{split}
\end{equation}
即
\begin{equation}
|G(\Delta t,\sigma)|^2=1-4a^2 r^2(1-a^2r^2)\sin^4\dfrac{\sigma h}{2}
\end{equation}
得稳定性条件为$|a|r\leq 1$.
\section{对流扩散方程}
考虑一个简单的对流扩散方程
\begin{equation}\label{equ:cf_dkfangc}
	\dfrac{\partial u}{\partial t}+a\dfrac{\partial u}{\partial x}=\nu\dfrac{\partial^2 u}{\partial x^2}
\end{equation}
其中a、$\nu$ 为常数,$\nu>0$,给定初值
\begin{equation}
	u(x,0)=g(x)
\end{equation}
构成了\emph{对流扩散方程}的初值问题,在我们的模型中,它是一个对流占优的扩散问题\upcite{Wang2011}。\par
\subsection{中心显式差分格式}
将方程~\eqref{equ:cf_dkfangc}~差分,有
\begin{equation}\label{equ:cf_dkfangct}
	\dfrac{u^{n+1}_j-u^{n}_{j}}{\tau}+a\dfrac{u^{n}_{j+1}-u^n_{j-1}}{2h}=\nu\dfrac{u^n_{j+1}-2u^n_j+u^n_{j-1}}{h^2}
\end{equation}
其截断误差为$O(\tau+h^2)$。\par
下面来分析差分格式~\eqref{equ:cf_dkfangct}~的稳定性,令
\begin{equation}
	\lambda = a\dfrac{\tau}{h},\quad\mu=\nu\dfrac{\tau}{h^2}
\end{equation}
则差分格式~\eqref{equ:cf_dkfangct}~改写为
\begin{equation}\label{equ:cf_dkmmm}
u^{n+1}_j=u^n_j-\frac{1}{2}\lambda(u^n_{j+1}-u^n_{j-1})+\mu(u^n_{j+1}-2u^n_j+u^n_{j-1})
\end{equation}\par
式~\eqref{equ:cf_dkmmm}~的增长因子为
\begin{equation}
	G(\tau,k)=1-2\mu(1-\cos kh)-i\lambda\sin kh
\end{equation}
模的平方为
\begin{equation}
\begin{aligned}
	|G(\tau,k)|^2 &= [1-2\mu(1-\cos kh)]^2+\lambda^2\sin^2 kh\\
				  &= 1-4\mu(1-\cos kh)+4\mu^2(1-\cos kh)^2+\lambda^2\sin^2 kh\\
				  &= 1-(1-\cos kh)[4\mu-4\mu^2(1-\cos kh)-\lambda^2(1+\cos kh)]
\end{aligned}				  
\end{equation}
差分格式稳定的充分条件为
\begin{equation}
4\mu-4\mu^2(1-\cos kh)-\lambda^2(1+\cos kh) \geq 0
\end{equation}
即
\begin{equation}
(2\lambda^2-8\mu^2)\dfrac{1-\cos kh}{2}+4\mu-2\lambda^2 \geq 0
\end{equation}
注意到$\dfrac{1}{2}(1-\cos kh)\in[0,1]$,上式应满足
\begin{equation}
(2\lambda^2-8\mu^2)+4\mu-2\lambda^2 \geq 0,\quad 4\mu-2\lambda^2 \geq 0
\end{equation}
由此得到差分格式~\eqref{equ:cf_dkfangct}~的稳定性限制为
\begin{equation}
\tau \leq \dfrac{2\nu}{a^2}
\end{equation}
\begin{equation}
\nu\dfrac{\tau}{h^2}=\dfrac{1}{2}
\end{equation}\par
\subsection{迎风差分格式}
根据中心显式差分格式的稳定性要求可知,当$\nu/a^2$比较小时,时间步长是比较小的.我们在一阶空间偏导数的离散中采用单边差商,
令$a>0$,得到方程~\eqref{equ:cf_dkfangc}~的迎风差分格式为
\begin{equation}\label{eq:cf_dkfangc_yf}
 \dfrac{u_j^{n+1}-u_j^n}{\tau}+a\dfrac{u_j^n-u_{j-1}^n}{h}=\mu\dfrac{u_{j+1}^n-2u_j^n+u_{j-1}^{n}}{h^2}
\end{equation}
容易看出,其截断误差为$O(\tau+h)$.\par
讨论其稳定性,将上式变为
\begin{equation}
 \dfrac{u_j^{n+1}-u_j^n}{\tau}+a\dfrac{u_{j+1}^n-u_{j-1}^n}{2h} = \left(\mu+\dfrac{ah}{2}\right)
 \dfrac{u_{j+1}^{n}-2u_j^n+u_{j-1}^n}{h^2}
\end{equation}
令$\bar{\mu}=\mu+\dfrac{ah}{2}$,可以得到式~\eqref{eq:cf_dkfangc_yf}~的稳定性条件为
\begin{equation*}
\begin{split}
  \tau\leq&\dfrac{2}{a^2}\bar{\mu} \\[0.7em]
  \bar{\mu}\dfrac{\tau}{h^2}\leq&\dfrac{1}{2}
\end{split}
\end{equation*}
注意到,第一个稳定性条件等价于$\dfrac{\tau}{\dfrac{2\mu}{a^2}+\dfrac{h}{a}}\leq 1$,而
\begin{equation*}
\dfrac{\tau}{\dfrac{2\mu}{a^2}+\dfrac{h}{a}}=\dfrac{2\mu\dfrac{\tau}{h^2}\left(\dfrac{ah}{2\mu}\right)^2}{1+\dfrac{ah}{2\mu}}
\end{equation*}
利用不等式
\begin{equation*}
\dfrac{\left(\dfrac{ah}{2\mu}\right)^2}{1+\dfrac{ah}{2\mu}}\leq 1+\dfrac{ah}{2\mu}
\end{equation*}
这样得到
\begin{equation*}
\dfrac{\tau}{\dfrac{2\mu}{a^2}+\dfrac{h}{a}}\leq 2\mu\dfrac{\tau}{h^2}\left(1+\dfrac{ah}{2\mu}\right)
\end{equation*}
因此差分格式的稳定性条件为
\begin{equation}
 \left(\mu+\dfrac{ah}{2}\right)\dfrac{\tau}{h^2}\leq\dfrac{1}{2}
\end{equation}\par
迎风差分格式主要用于对流占优的扩散问题,采用中心差分格式不能很好地得出结果,采用迎风格式可以计算出问题的近似解.
\subsection{隐式差分格式}
考虑到隐性差分格式带来的诸多好处(如无条件稳定),给出迎风差分格式的隐式格式为
\begin{equation}\label{eq:04_yfcf_ys}
 \dfrac{u_j^{n+1}-u_j^n}{\tau}+a\dfrac{u_{j+1}^{n+1}-u_{j-1}^{n+1}}{2h}=\nu\sigma\dfrac{u_{j+1}^{n+1}
 -2u_j^{n+1}+u_{j-1}^{n+1}}{h^2}
\end{equation}
此格式是无条件稳定的.\par
现在考虑隐式差分格式的解法,将式~\eqref{eq:04_yfcf_ys}~化为
\begin{equation*}
-\left(\dfrac{a}{2h}+\dfrac{\nu\sigma}{h^2}\right)u_{j-1}^{n+1}+\left(\dfrac{a}{2h}-\dfrac{\nu\sigma}{h^2}\right)u_{j+1}^{n+1}
+\left(\dfrac{1}{2}+\dfrac{2\sigma}{h^2}\right)u_j^{n+1}=\dfrac{1}{2}u_j^n
\end{equation*}
令
\begin{equation*}
\begin{aligned}
A_j &= -\dfrac{a}{2h}-\dfrac{\nu\sigma}{h^2} & \qquad & B_j = \dfrac{1}{2}+\dfrac{2\sigma}{h^2} \\[0.6em]
C_j &= \dfrac{a}{2h}-\dfrac{\nu\sigma}{h^2}  &  \qquad & D_j = \dfrac{1}{2}   \\
\end{aligned}
\end{equation*}
则方程可以写为
\begin{equation}\label{eq:04_yfcf_ys_b}
 A_j u_{j-1}^{n+1} + B_j u_j^{n+1} + C_j u_{j+1}^{n+1} = D_j
\end{equation}
其中,$j=1,2,\ldots,J-1$.我们令
\begin{equation}\label{eq:04_yfcf_ys_bj}
 u_0^{n+1}=0,\quad u_J^{n+1}=0
\end{equation}
为边界条件,方程~\eqref{eq:04_yfcf_ys_b}~和方程~\eqref{eq:04_yfcf_ys_bj}~构成系数为三对角矩阵的方程组,我们可以利用
追赶法求解这一方程组.\par
我们给出三对角矩阵存在Crout分解的充分条件,有
\begin{Theorem}[Crout分解的充分条件]
对于三对角矩阵
\begin{equation*}
A=\begin{bmatrix}
   b_1 & c_1 \\
   a_2 & b_2 & c_2 \\
       & a_3 & b_3    & c_3 \\
       &     & \ddots & \ddots  & \ddots            \\
       &     &        & a_{n-1} & b_{n-1} & c_{n-1} \\
       &     &        &         &   a_n   & b_n     \\
  \end{bmatrix}
\end{equation*}
存在Crout分解的充分条件为$a_i\not=0,(i=2,3,\ldots,n),c_1\not=0$,且
\begin{equation*}
\begin{split}
|b_1|&>|c_1|>0 \\
|b_i|&>|a_1|+|c_i|,(i=2,3,\ldots,n-1) \\
|b_n|&>|a_n|   \\
\end{split}
\end{equation*}
\end{Theorem}\par
方程~\eqref{eq:04_yfcf_ys_b}~和方程~\eqref{eq:04_yfcf_ys_bj}~的追赶法
求解充分条件为
\begin{equation}
\begin{cases}
|B_1|>|C_1|>0,\\
|B_j|\geq |A_j|+|C_j|,\quad A_j C_j\not=0,\quad j=2,\ldots,J-2,\\
|B_{J-1}|>|A_{J-1}| \\
\end{cases}
\end{equation}\par
利用矩阵的Crout分解解方程组,差分后得到的方程组可以写为
\begin{equation}~\label{eq:04_yfcf_jgmt}
\begin{bmatrix}
   B_1 & C_1 \\
   A_2 & B_2 & C_2 \\
       &  & \ddots    &  \\
       &     & A_j &  B_j    & C_j          \\
       &     &        &  & \ddots     \\
       &     &        &        &   A_{J-1}   & B_{J-1}     \\
\end{bmatrix}
\begin{bmatrix}
u_0^{n+1} \\
u_1^{n+1} \\
\vdots    \\
\vdots    \\
u_{J-2}^{n+1} \\
u_{J-1}^{n+1} \\
\end{bmatrix}
=
\begin{bmatrix}
D_0 \\
D_1\\
\vdots    \\
\vdots    \\
D_{J-2} \\
D_{J-1} \\
\end{bmatrix}
\end{equation}
对式~\eqref{eq:04_yfcf_jgmt}~中的矩阵系数$A$进行Court分解,得
\begin{multline}
\begin{bmatrix}
   B_1 & C_1 \\
   A_2 & B_2 & C_2 \\
       &  & \ddots    &  \\
       &     & A_j &  B_j    & C_j          \\
       &     &        &  & \ddots     \\
       &     &        &        &   A_{J-1}   & B_{J-1}     \\
\end{bmatrix}=\\
\begin{bmatrix}
l_1 & \\
m_2 & l_2 \\
    & m_3 & l_3 \\
    &     & \ddots & \ddots & \\
    &     &        & m_{J-2} & l_{J-1} \\
    &	  &	   &	     &  m_{J-1} & l_{J-1} \\
\end{bmatrix}
\begin{bmatrix}
1 & u_0       \\
  &  1  & u_1 \\
  &     &   1  &  u_2 \\
  &     &      &   \ddots & \ddots \\
  &     &      &          &    1   & u_{J-1} \\
  &     &      &          &        &      1   \\
\end{bmatrix}
\end{multline}
其中,
\begin{equation}
\begin{cases}
m_i=a_i,\quad (i=2,3,\ldots,J-1) \\
l_1=b_1,u=c_1/l_1 \\
l_i=b_i-m_iu_{i-1},\quad (i=2,3,\ldots,J-1) \\
u_i=c_i/l_i,\quad (i=2,3,\ldots,J-2) 
\end{cases}
\end{equation}\par
利用
\begin{equation}
\begin{bmatrix}
l_1 & \\
m_2 & l_2 \\
    & m_3 & l_3 \\
    &     & \ddots & \ddots & \\
    &     &        & m_{J-2} & l_{J-1} \\
    &	  &	   &	     &  m_{J-1} & l_{J-1} \\
\end{bmatrix}
\begin{bmatrix}
y_0 \\
y_1 \\
y_2 \\
\vdots \\
y_{J-2} \\
y_{J-1} \\
\end{bmatrix}=
\begin{bmatrix}
 d_0 \\
 d_1 \\
 d_2 \\
 \vdots \\
 d_{J-2} \\
 d_{J-1} \\
\end{bmatrix}
\end{equation}
其中,
\begin{equation}
\begin{cases}
y_1=d_1/l_1 \\
y_i=(d_i-m_iy_{i-1})/l_i,\quad (i=2,3,\ldots,J-1)
\end{cases}
\end{equation}
解得$Y^{\mathrm{T}}$.\par
将$Y^{\mathrm{T}}$代入
\begin{equation}
\begin{bmatrix}
1 & u_0       \\
  &  1  & u_1 \\
  &     &   1  &  u_2 \\
  &     &      &   \ddots & \ddots \\
  &     &      &          &    1   & u_{J-1} \\
  &     &      &          &        &      1   \\
\end{bmatrix}
\begin{bmatrix}
x_0 \\
x_1 \\
x_2 \\
\vdots \\
x_{J-2} \\
x_{J-1} \\
\end{bmatrix}=
\begin{bmatrix}
y_0 \\
y_1 \\
y_2 \\
\vdots \\
y_{J-2} \\
y_{J-1} \\ 
\end{bmatrix}
\end{equation}
其中,
\begin{equation}
\begin{cases}
x_n=y_n \\
x_i=y_i-u_ix_{i+1},\quad (i=J-1,J-2,\ldots,1)
\end{cases}
\end{equation}
就解得了$X^{\mathrm{T}}$.
\section{多维方程}
多维方程是一维问题的推广,但是在求解的过程中会遇到诸多的问题.一维的差分格式直接拓展到多维上会导致稳定性条件变得严苛.若
采用隐式格式会无法使用追赶法从而导致求解困难.所以,在本节我们重点讨论多维方程差分格式的建立过程.
\subsection{二维扩散方程}
考虑二维上的扩散方程的初边值问题
\begin{equation}\label{eq:04_ew_q}
\begin{cases}
\dfrac{\partial u}{\partial t}=a\left(\dfrac{\partial^2 u}{\partial x^2}+\dfrac{\partial^2 u}{\partial y^2}\right),& 0<x,y<1,\quad t>0, \\
u(x,y,0)=u_0(x,y),& 0<x,y<1 \\
u(0,y,t)=u(1,y,t)=u(x,0,t)=u(x,1,t)=0,& t>0
\end{cases}
\end{equation}
其中,~$a$为常数.\par
为了讨论差分格式,我们先将定义域
\begin{equation*}
D=\{(x,y,t)|0\leq x,y\leq 1,t\geq 0\}
\end{equation*}
划分网格
\begin{equation}
\begin{split}
D_h=\{(x_j,y_l,t_n)|x_j=jh,j=0,1,\ldots,J,Jh=1 \\
y_l=dh,l=0,1,\ldots,J;\quad t_n=n\tau,n\geq0\},
\nonumber
\end{split}
\end{equation}
其中,~$\tau$和$h$分别为时间步长和空间步长.我们考虑取$x$方向和$y$方向上的步长相等.引入记号
\begin{equation*}
\begin{split}
\delta_x^2 u_{jl}^n =& u_{j+1,l}^n-2u_{jl}^n+u_{j-1,l}^n, \\
\delta_y^2 u_{jl}^n =& u_{j,l+1}^n-2u_{jl}^n+u_{j,l-1}^n.
\end{split}
\end{equation*}
现在介绍Peaceman-Rachford格式(P--R格式),即
\begin{equation}\label{eq:04_ew_pr}
\begin{cases}
\dfrac{u_{jl}^{n+\frac{1}{2}}-u_{jl}^n}{\dfrac{\tau}{2}}=a\dfrac{1}{h^2}(\delta_x^2u_{jl}^{n+\frac{1}{2}}+
\delta_y^2 u_{jl}^n),\\
\dfrac{u_{jl}^{n+1}-u_{jl}^{n+\frac{1}{2}}}{\dfrac{\tau}{2}}=a\dfrac{1}{h^2}(\delta_x^2u_{jl}^{n+\frac{1}{2}}+
\delta_y^2u_{jl}^{n+1}). 
\end{cases}
\end{equation}\par
可以看到,计算$u_{jl}^{n+1}$由两步组成,在每一个方向上是隐式差分的.所以多次利用追赶法来求解.\par
我们来分析P--R格式的精确度.将式~\eqref{eq:04_ew_pr}~两式相加,得
\begin{equation}\label{eq:04_ew_pr_plus}
\dfrac{u_{jl}^{n+1}-u_{jl}^n}{\dfrac{\tau}{2}}=\dfrac{2a}{h^2}\delta_x^2u_{jl}^{n+\tfrac{1}{2}}
+\dfrac{a}{h^2}\delta_y^2(u_{jl}^{n+1}+u_{jl}^n).
\end{equation}
再将式~\eqref{eq:04_ew_pr}~两式相减,有
\begin{equation}
4u_{jl}^{n+\frac{1}{2}}=2(u_{jl}^{n+1}+u_{jl}^n)-\dfrac{\tau a}{h^2}\delta_y^2(u_{jl}^{n+1}+u_{jl}^n)
\end{equation}
代入式~\eqref{eq:04_ew_pr_plus}~中,得
\begin{equation}\label{eq:04_ew_pr_zl}
\left(1+\dfrac{1}{4}\dfrac{\tau^2 a^2}{h^4}\delta_x^2\delta_y^2\right)\dfrac{u_{jl}^{n+1}-u_{jl}^n}{\tau}
=\dfrac{a}{h^2}(\delta_x^2+\delta_y^2)\dfrac{u_{jl}^{n+1}-u_{jl}^n}{2}
\end{equation}
设$u(x,t)$为方程~\eqref{eq:04_ew_q}~的解析解,设$u(x,t)$关于$t$三次连续可微,关于$x$,$y$连续四次可微,利用Taylor级数
展开
\begin{equation*}
\begin{split}
&\left(1+\dfrac{1}{4}\dfrac{\tau^2 a^2}{h^4}\delta_x^2\delta_y^2\right)\dfrac{u(x_j,y_l,t_{n+1})-
u(x_j,y_l,t_n)}{\tau} \\[0.3\baselineskip]
-& \dfrac{a}{h^2}(\delta_x^2+\delta_y^2)\dfrac{u(x_j,y_l,t_{n+1})+u(x_j,y_l,t_n)}{2} \\[0.3\baselineskip]
=& O(\tau^2+h^2).
\end{split}
\end{equation*}
因此P--R格式是具有二阶精度的.\par
我们讨论P--R格式的稳定性,将式~\eqref{eq:04_ew_pr_zl}~整理,有
\begin{equation}\label{eq:04_ew_pr_wd}
 \left(1-\dfrac{a\lambda}{2}\delta_x^2\right)\left(1-\dfrac{a\lambda}{2}\delta_y^2\right)u_{jl}^{n+1}
 =\left(1+\dfrac{a\lambda}{2}\delta_x^2\right)\left(1+\dfrac{a\lambda}{2}\delta_y^2\right)u_{jl}^n
\end{equation}
其中,$\lambda=\tau/h^2$,求增长因子为
\begin{equation*}
G(\tau,k)=\cfrac{\left(1-2a\lambda\sin^2\dfrac{k_1h}{2}\right)\left(1-2a\lambda\sin^2\dfrac{k_2h}{2}\right)}
{\left(1+2a\lambda\sin^2\dfrac{k_1h}{2}\right)\left(1+2a\lambda\sin^2\dfrac{k_2h}{2}\right)}
\end{equation*}
显然,对于任意的$\lambda$有$|G(\tau,k)\leq 1|$,所以P--R格式是稳定的.\par
为了将二维的问题求解方法推广到高维,我们来推导Dougals格式,将式~\eqref{eq:04_ew_pr_wd}~变形为
\begin{equation}
\left(1-\dfrac{a}{2}\lambda\delta_x^2\right)\left(1-\dfrac{a}{2}\lambda\delta_y^2\right)
\dfrac{u_{jl}^{n+1}-u_{jl}^{n}}{\tau}=\dfrac{a}{h^2}(\delta_x^2+\delta_y^2)u_{jl}^n.
\end{equation}
将上式分解为
\begin{equation}\label{eq:04_ew_dg_1}
\left(1-\dfrac{a}{2}\lambda\delta_x^2\right)\dfrac{\u_{jl}^{n+\frac{1}{2}}-u_{jl}^n}{\tau}
=\dfrac{a}{h^2}(\delta_x^2+\delta_y^2)u_{jl}^n
\end{equation}
及
\begin{equation}
\left(1-\dfrac{a}{2}\lambda\delta_y^2\right)\dfrac{u_{jl}^{n+1}-u_{jl}^{n}}{\tau}=
\dfrac{u_{jl}^{n+\frac{1}{2}}-u_{jl}^n}{\tau}.
\end{equation}
将上式变为
\begin{equation}\label{eq:04_ew_dg_2}
\dfrac{u_{jl}^{n+1}-u_{jl}^{n+\frac{1}{2}}}{\dfrac{\tau}{2}}=\dfrac{a}{h^2}\delta_y^2(u_{jl}^{n+1}
-u_{jl}^n).
\end{equation}
称格式~\eqref{eq:04_ew_dg_1}~和格式~\eqref{eq:04_ew_dg_2}~称为Douglas格式.
\subsection{三维扩散方程}
考虑三维扩散方程
\begin{equation}
\dfrac{\partial u}{\partial t}=a\left(\dfrac{\partial^2 u}{\partial x^2}+
\dfrac{\partial^2 u}{\partial y^2}+\dfrac{\partial^2 u}{\partial z^2}\right)
\end{equation}
相应地确定它的定解条件.
考虑二维Peaceman-Rachford格式的推广,设$\Delta x=\Delta y=\Delta z$,~P--R格式可以写为
\begin{equation}\label{eq:04_3d_pr}
\begin{split}
\left(1-\dfrac{a\lambda}{3}\delta_x^2\right)u_{jlm}^{n+\frac{1}{3}} &= 
\left(1+\dfrac{d\lambda}{3}\delta_y^2+\dfrac{a\lambda}{3}\delta_z^2\right)u_{jlm}^n, \\
\left(1-\dfrac{a\lambda}{3}\delta_y^2\right)u_{jlm}^{n+\frac{1}{3}} &= 
\left(1+\dfrac{d\lambda}{3}\delta_x^2+\dfrac{a\lambda}{3}\delta_z^2\right)u_{jlm}^{n+\frac{1}{3}}, \\
\left(1-\dfrac{a\lambda}{3}\delta_z^2\right)u_{jlm}^{n+\frac{1}{3}} &= 
\left(1+\dfrac{d\lambda}{3}\delta_x^2+\dfrac{a\lambda}{3}\delta_y^2\right)u_{jlm}^{n+\frac{2}{3}},
\end{split}
\end{equation}
此格式的截断误差为$O(\Delta t+3h^3)$,误差过大,不采用.\par
将式~\eqref{eq:04_3d_pr}~变为
\begin{multline}
\left(1-\dfrac{a\lambda}{3}\delta_x^2\right)\left(1-\dfrac{a\lambda}{3}\delta_y^2\right)\left(1-\dfrac{a\lambda}{3}\delta_z^2\right)
u_{jlm}^{n+1}=\\
\left(1+\dfrac{a\lambda}{3}\delta_x^2\right)
\left(1+\dfrac{a\lambda}{3}\delta_y^2\right)
\left(1+\dfrac{a\lambda}{3}\delta_z^2\right)
u_{jlm}^n
\end{multline}
的截断误差为$O(\tau^2+3h^2)$.\par
注意到,上式等价于
\begin{multline}
\left(1-\dfrac{a\lambda}{3}\delta_x^2\right)\left(1-\dfrac{a\lambda}{3}\delta_y^2\right)\left(1-\dfrac{a\lambda}{3}\delta_z^2\right)
(u_{jlm}^{n+1}-u_{jlm}^n)=\\
a\lambda(\delta_x^2+\delta_y^2+\delta_z^2)u_{jlm}^n+\dfrac{(a\lambda)^3}{4}\delta_x^2\delta_y^2\delta_z^2
u_{jlm}^n
\end{multline}
忽略高阶项,有
\begin{equation}
\begin{cases}
\left(1-\dfrac{a\lambda}{3}\delta_x^2\right)\Delta u^*=a\lambda(\delta_x^2+\delta_y^2+\delta_z^2)u_{jlm}^n \\[15pt]
\left(1+\dfrac{a\lambda}{3}\delta_y^2\right)\Delta u^{**} = \Delta u^* \\[15pt]
\left(1+\dfrac{a\lambda}{3}\delta_z^2\right)\Delta u = \Delta u^{**} \\[15pt]
\Delta u = u_{jlm}^{n+1}-u_{jlm}^n
\end{cases}
\end{equation}
为无条件稳定的二阶精度格式.
\subsection{三维对流扩散反应方程}
现在我们开始讨论在空间上建立和求解数学模型,根据运动控制方程,我们得到物质在空间上的分布为
% \begin{equation}
%  \dfrac{\partial u}{\partial t}=a\left(\dfrac{\partial^2 u}{\partial x^2}+\dfrac{\partial^2 u}{\partial y^2}
%  +\dfrac{\partial^2 u}{\partial z^2}\right)-b\left(\dfrac{\partial u}{\partial x}+\dfrac{\partial u}{\partial y}
%  +\dfrac{\partial u}{\partial z}\right)+cu+d
% \end{equation}
\begin{equation}\label{eq:04_3d_a}
R_d\dfrac{\partial C}{\partial t}+v_i\dfrac{\partial C}{\partial x_i}=\dfrac{\partial}{\partial x_i}(D_{ij}\dfrac{\partial C}
{\partial x_j})-\mu C+\lambda,\quad (i,j=1,2,3)
\end{equation}\par
我们考虑对上式进行变换,将直角坐标系变换为正交曲线坐标系,以溶质位移作为
主值方向,式~\eqref{eq:04_3d_a}变为
\begin{equation}
R_d\dfrac{\partial C}{\partial t}+u\dfrac{\partial C}
{\partial S}=\dfrac{\partial}{\partial S}(D_L\dfrac{\partial C}{\partial S})
+\dfrac{\partial}{\partial R}(D_R\dfrac{\partial C}{\partial R})
+\dfrac{\partial}{\partial T}(D_T\dfrac{\partial C}{\partial T})-\mu C+\lambda
\end{equation}
其中,~$S$为流线方向,~$R$,$T$为与$S$相交的方向.~$D_L$,$D_R$和$D_T$为三个方向的系数.\par
我们令$D_L$,~$D_R$和$D_T$为常数,上式变为
\begin{equation}
R_d\dfrac{\partial C}{\partial t}+u\dfrac{\partial C}
{\partial S}=D_L\dfrac{\partial^2 C}{\partial S^2}+D_R\dfrac{\partial^2 C}{\partial R^2}+
D_T\dfrac{\partial^2 C}{\partial T^2}-\mu C+\lambda
\end{equation}
采用\emph{算子分裂法},得
\begin{align}
&\dfrac{1}{3}R_d\dfrac{\partial C}{\partial t}=-\mu C+\lambda , & n\Delta t&\leq t\leq(n+\dfrac{1}{3})\Delta t \label{eq:04_3d_p1}\\
&\dfrac{1}{3}R_d\dfrac{\partial C}{\partial t}+ u\dfrac{\partial C}{\partial S}=D_L\dfrac{\partial^2 C}{\partial S^2}, & (n+\dfrac{1}{3})\Delta t&\leq t\leq (n+\dfrac{2}{3})\Delta t \label{eq:04_3d_p2}\\
&\dfrac{1}{3}R_d\dfrac{\partial C}{\partial t}=D_R\dfrac{\partial^2 C}{\partial R^2}+D_T\dfrac{\partial^2 C}{\partial T^2},& (n+\dfrac{2}{3})\Delta t&\leq t\leq(n+1)\Delta t \label{eq:04_3d_p3}
\end{align}
我们采用常数变易法来求解式~\eqref{eq:04_3d_p1},该方程为线性一阶常微分方程,初始条件为$C^{n+1/3}|_{t=t_n}=C^n$.\par
应用常数变易法求得~\eqref{eq:04_3d_p1}的解为
\begin{equation}
\begin{cases}
C_{i,j,k}^{n+1/3}=(C_{i,j,k}^n-\dfrac{\lambda}{\mu})\exp(-\mu\Delta t/Rd)+\dfrac{\lambda}{\mu},& \quad (\mu\not=0) \\[0.8em]
C_{i,j,k}^{n+1/3}=C_{i,j,k}^n+\dfrac{\lambda\Delta t}{Rd},&\quad (\mu=0)
\end{cases}
\end{equation}\par
我们采用匹配伪弥散系数法来求解~\eqref{eq:04_3d_p2}~,以避免数值弥散和震荡.\par
与~\eqref{eq:04_3d_p2}~式对应的纯对流方程为
\begin{equation}\label{eq:04_3d_p2c}
\dfrac{1}{3} R_d\dfrac{\partial C}{\partial t}+u\dfrac{\partial C}{\partial S}=0,\quad (n+\dfrac{1}{3})\Delta t\leq t\leq (n+\dfrac{2}{3})\Delta t
\end{equation}
采用差分法,得
\begin{align}
\dfrac{\partial C}{\partial t} &= 3\theta\dfrac{C_{i,j,k}^{n+2/3}-C_{i,j,k}^{n+1/3}}{\Delta t}+3(1-\theta)\dfrac{C_{i,j,k}^{n+2/3}-C_{i,j,k}^{n+1/3}}{\Delta t} \label{eq:04_3d_p2_cf1}\\[0.6em]
\dfrac{\partial C}{\partial S} &= \left[\dfrac{C_{i,j,k}^{n+2/3}-C_{i,j,k}^{n+1/3}}{2}-\dfrac{C_{i,j,k}^{n+2/3}-C_{i,j,k}^{n+1/3}}{2}\right] / \Delta S_i \label{eq:04_3d_p2_cf2}
\end{align}
将式~\eqref{eq:04_3d_p2_cf1}~和式~\eqref{eq:04_3d_p2_cf2}~代入式~\eqref{eq:04_3d_p2c}~,整理得
\begin{equation}\label{eq:04_3d_p2zl}
C_{i+1,j,k}^{n+2/3}=K_1C_{i,j,k}^{n+1/3}+K_2C_{i,j,k}^{n+2/3}+K_3C_{i+1,j,k}^{n+1/3}
\end{equation}
式中,
\begin{equation*}
K_1=\dfrac{Cr+2\theta}{2+Cr-2\theta},\quad K_2=\dfrac{Cr-2\theta}{2+Cr-2\theta},\quad K_3=\dfrac{2-Cr-2\theta}{2+Cr-2\theta}
\end{equation*}
其中,~$Cr=u\Delta t/(R_d\Delta S_i)$.对上式进行Taylor级数展开到二阶,得到一个对流弥散方程.其弥散
系数来自二阶阶段误差,其表达式为
\begin{equation*}
D_N=u\Delta S_i(0.5-\theta)
\end{equation*}
令$D_L=D_N$,得$\theta=0.5-(1/Pe)$,其中$Pe=u\Delta S_i/D_L$.\par
考虑式~\eqref{eq:04_3d_p2zl}~的稳定性,要求满足
\begin{align}
2 & \leq Pe \leq \infty\\
1-\dfrac{2}{Pe}&\leq Cr\leq 1+\dfrac{2}{Pe}
\end{align}
我们看到式~\eqref{eq:04_3d_p2zl}~是无条件稳定的,具有二阶的精度.\par
再利用交替方向差分法来求解~\eqref{eq:04_3d_p3},其交替方向差分格式为
\begin{align}
R_d\dfrac{C_{i,j,k}^{n+5/6}-C_{i,j,k}^{n+2/3}}{\Delta t} &=
D_R\delta_R^2 C_{i,j,k}^{n+5/6}+D_{\tau}\delta^2_{\tau}C_{i,j,k}^{n+2/3} \label{eq:04_3d_p3adi1}\\
R_d\dfrac{C_{i,j,k}^{n+1}-C_{i,j,k}^{n+5/6}}{\Delta t}&=D_R\delta_R^2 C_{i,j,k}^{n+5/6}+D_{\tau}\delta_{\tau}^2+C_{i,j,k}^{n+1} \label{eq:04_3d_p3adi2} 
\end{align}
其中,
\begin{equation*}
\begin{split}
\delta_R^2 C_{i,j,k} =& [\dfrac{C_{i,j-1,k}}{(R_{j+1}-R_{j-1})(R_j-R_{j-1})}
-\dfrac{C_{i,j,k}}{(R_{j+1}-R_j)(R_j-R_{j-1})} \\
&+\dfrac{C_{i,j+1}}{(R_{j+1,k}-R_{j-1})(R_{j+1}-R_j)}] \\
\delta_T^2 C_{i,j,k} =& [\dfrac{C_{i,j,k-1}}{(T_{k+1}-T_{k-1})(T_k-T_{k-1})}-
\dfrac{C_{i,j,k}}{(T_{k+1}-T_k)(T_k-T_{k-1})} \\
&+\dfrac{C_{i,j,k+1}}{(T_{k+1}-T_{k-1})(T_{k+1}-T_k)}]
\end{split}
\end{equation*}
方程~\eqref{eq:04_3d_p3adi1}~和方程~\eqref{eq:04_3d_p3adi2}~为
三对角线性方程组,用追赶法求解.\par
这样,按时间顺序求解~\eqref{eq:04_3d_p1}~,~\eqref{eq:04_3d_p2}~,~\eqref{eq:04_3d_p3}~,
就可以求得$n+1$各点的浓度.
\section{本章小结}
本章主要讨论了对各类偏微分方程的数值求解法.首先,对常系数扩散方程、对流方程和对流扩散方程进行差分分解,讨论了常用的差分格式并
证明了它们的稳定性和准确度.然后,采用了算子分裂法对求解时间进行分裂,将三维的对流扩散反应方程进行分解得到了分数步的差分格式,
将对流扩散反应方程分解成对流方程、扩散方程和反应方程,分别对它们进行求解.
