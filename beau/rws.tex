\pagestyle{empty}
{\pagestyle{empty}
\begin{center}
\zihao{3}\heiti 毕业设计任务书
\end{center}

\setlength{\baselineskip}{25pt}
\noindent 设计(论文)题目:\uline{\hfill 微生物土壤运移模型的求解及仿真软件编制 \hfill} \\
\noindent 学院:\uline{\qquad生命科学与技术学院\qquad}\quad 专业:\uline{\quad制药工程\quad}\quad 班级:\uline{\hfill制药0901\hfill} \\
\noindent 学生:\uline{\hfill 陆秋文 \hfill}\quad 指导教师(含职称):\uline{\hfill 周延(讲师) \hfill} \quad 专业负责人:\uline{\hfill 郑国钧\hfill} \\
\noindent 1.设计(论文)的主要任务及目标\par
\begin{asparaenum}[(1)]
 \item 收集与模型相关的微生物系数;
 \item 使用软件解出微生物运动的微分方程;
 \item 输入实验数据,进行模型验证;
 \item 制作仿真软件,在界面显示微生物运动实际情况.
\end{asparaenum}
2.设计(论文)的基本要求和内容
\begin{asparaenum}[(1)]
 \item 论文格式正确,行文具有逻辑性,论证严密,表达精确;
 \item 实验设计能够达到预期的目标,实验步骤正确可靠,实验数据准确.
 \item 数学推导严密,论证充分.
\end{asparaenum}
3.主要参考文献
\begin{publist}
\item  张瑞玲.甲基叔丁基醚的生物降解机理与微生物在地下水中的迁移[D].天津:天津大学,2007.
\item  刘法,缪国庆,梁昆淼.数学物理方法(第四版)[M].北京: 高等教育出版社, 2010.
\item  张文生.科学计算中的偏微分方程有限差分法 [M]. 中国科学院研究生院教材.北京:高等教育出版社,2006.
\end{publist}
4.进度安排\vspace{\baselineskip}
\begin{center}
\begin{tabularx}{14cm}{|c|X|c|}
\hline
   & \centering 设计(论文)各阶段名称	& 起止日期 \\
\hline
 1 & 文献查阅与相关知识学习,技术准备   & 2012年12月 \\
\hline
 2 & 完成一维方程的求解与仿真          & 2013年1、2月 \\
 \hline
 3 & 完成三维方程的建模和边界条件的确定  & 2013年3、4月 \\
 \hline
 4 & 三维方程求解算法的研究、完成论文    & 2013年5、6月 \\
 \hline
\end{tabularx}
\end{center}
}
