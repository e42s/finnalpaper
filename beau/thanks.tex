\chapter*{致谢}
在这篇论文即将完成之际,首先要感谢我的导师周延老师.周老师在教学科研上严谨认真,学识渊博.在我完成毕业设计期间,尽心指导,亲力亲为,
不仅在理论上提出了诸多建议意见,在细节上也帮忙订正了许多错误,令我获益菲浅.不仅如此,周老师待人温和,对我非常关心,
在毕业期间对我个人前途和发展等诸多方面提出了很多建议,令我十分感动.非常感谢周老师对我的指导和帮助.\par
同时,感谢实验室的耿婧师姐.耿师姐在关于本课题的实验方面给了很多的指导,对研究过程中很多方面都给出了建议.\par
感谢理学院数学系黄晋阳教授在求解解析解过程中给予的帮助.\par
感谢我的本科生导师,同时也是我的研究生导师孙洪程高级工程师.孙洪程老师在我本科学习期间,对我的学业、生活等诸多
方面都给予了充分的指导与关心,十分照顾.尤其是他在自动化相关工程技术上丰富的经验和学识对我影响很深.不仅如此,
孙老师为人谦和,待人亲切,对学生宽容体谅,教学认真负责,一丝不苟.希望在今后研究生阶段能够得到孙老师更多的指导.\par
感谢辽阳市第一高级中学的张波老师和辽阳市新竹学校的乔立男校长、孙丽红老师和其他关心我成长的诸位老师.\par
感谢北京化工大学广播台的各位同事.与你们在一起生活、学习、工作,感到十分的快乐和幸福.和你们在一起的时光,是我一生
难以忘怀的珍贵记忆.非常感谢你们与我走过四年的大学生活,有了你们,大学变得丰富多彩而有意义.\par
感谢我的父母、祖父母和家人.多年来,你们对我的养育和教导使我取得了今天的成绩.无论客观条件发生了多么
大的变化,无论生活有多么的困难和坎坷,你们对我的关心和爱护始终如一,从来没有发生过变化和动摇.正所谓``谁言寸草心,报得三春晖'',
养育之恩无以为报,只有在今后的生活中更加努力,尽到自己的责任,才是对你们最好的报答和补偿.\par
最后,再一次感谢所有关心我的人.
\par
\begin{flushright}
\begin{minipage}[t]{7em}
\centering\kaishu
陆秋文\\
2013年6月
\end{minipage}
\end{flushright}


\chapter*{作者及导师简介}
\noindent 陆秋文,北京化工大学生命科学与技术学院2009级制药工程专业毕业生.\par
\noindent 2013年起在北京化工大学信息科学与技术学院攻读硕士学位.\par
\noindent E-mail:~luqiuwen@gmail.com\par

\vspace{3em}
\noindent 周延,北京化工大学生命科学与技术学院制药工程专业讲师\par
\noindent 研究方向:微生物\par
\noindent E-mail:~zhouyan@mail.buct.edu.cn