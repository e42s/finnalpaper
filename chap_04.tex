\chapter{数值解}
对于大部分的偏微分方程模型问题来说,解得它们的解析解是比较困难的.在大部分的情况下,我们不能也没有必要求得它们的解析解.
因此,研究它们的数值解的求解方法是很有必要的.\par
在这一章中,我们同样从简单的方程开始解,建立数值解的求解方法的基本体系,然后我们再研究复杂模型的求解方法.
\section{常系数扩散方程}
考虑常系数扩散方程
\begin{equation}\label{eq:04_cxsks_o}
\begin{cases}
 \dfrac{\partial u}{\partial t}=a\dfrac{\partial^2 u}{\partial x^2} \\
 u(x,0)=g(x),\quad x\in\mathbb{R}
\end{cases}
\end{equation}
构成了初值问题,其向前差分格式为
\begin{gather}
 \dfrac{u_{j}^{n+1}-u_j^n}{\tau}-a\dfrac{u_{j+1}^n-2u_j^n+u_{j-1}^n}{h^2}=0,\label{eq:04_cxsks_xq}\\
 u_j^0 = g(x_j)
\end{gather}
其截断误差为$O(\tau+h^2)$.\par
考虑它的向后差分格式
\begin{equation}\label{eq:04_cxsks_xh}
 \dfrac{u_j^n-u_j^{n-1}}{\tau}-a\dfrac{u_{j+1}^n-2u_j^n+u_{j+1}^n}{h^2}=0
\end{equation}
其截断误差也是$O(\tau+h^2)$.
\subsection{加权隐式格式}
将式~\ref{eq:04_cxsks_xq}~改写为
\begin{equation}\label{eq:04_cxsks_xq_gx}
 \dfrac{u_j^n-u_j^{n-1}}{\tau}-a\dfrac{u_{j+1}^{n-1}-2u_{j}^{n-1}+u_{j-1}^{n-1}}{h^2}=0
\end{equation}
在式~\ref{eq:04_cxsks_xq}~乘以$\theta$,用$(1-\theta)$乘以~\ref{eq:04_cxsks_xq_gx},得到差分格式
\begin{equation}\label{eq:04_cxsks_js}
 \dfrac{u_j^n-u_j^{n-1}}{\tau}-a\left[\theta\dfrac{u_{j+1}^n-2u_j^n+u_{j-1}^n}{h^2}
 +(1-\theta)\dfrac{u_{j+1}^{n-1}-2u_j^{n-1}+u_{j-1}^{n-1}}{h^2}\right]=0
\end{equation}
其中,~$0\leq\theta\leq 1$,这种差分格式称为加权差分格式,将其改写为易于计算的格式
\begin{multline}
 -a\lambda\theta_{j+1}^n+(1+2a\lambda\theta)u_j^n-a\lambda\theta u_{j-1}^n = \\
 a\lambda(1-\theta)u_{j+1}^{n-1}+[1-2a\lambda(1-\theta)]u_j^{n-1}+a\lambda(1-\theta)u_{j-1}^{n-1}
\end{multline}
其中,$\lambda=\tau/h^2$,我们求差分格式~\ref{eq:04_cxsks_js}~的截断误差,设$u(x,t)$是方程~\ref{eq:04_cxsks_o}~的
充分光滑的解,在$(x_j,t_n)$处进行Taylor级数展开得
\begin{equation*}
E=a\left(\dfrac{1}{2}-\theta\right)\tau\left[\dfrac{\partial^3 u}{\partial x^2}{\partial t}\right]_j^n
+O(\tau^2+h^2)
\end{equation*}
当$\theta\not=1/2$时,其截断误差为$O(\tau+h^2)$.当$\theta=1/2$时,其截断误差为$O(\tau^2+h^2)$.\par
采用Fourier方法分析差分格式~\ref{eq:04_cxsks_js}~的稳定性,求得其增长因子为
\begin{equation}
 G(\tau,k)=\dfrac{1-4(1-\theta)a\lambda\sin^2\dfrac{kh}{2}}{1+4\theta a\lambda\sin^2\dfrac{kh}{2}}
\end{equation}
我们给出用于判断差分格式稳定性的\textbf{von Neumann}~定理,即
\begin{Theorem}[von Neumann定理]
差分格式
\begin{equation}
	u^{n+1}_j=C(x_j,\tau)u^n_j
\end{equation}
稳定的必要条件是当$\tau\leq\tau_0$,$n\tau\leq T$,对所有的k有
\begin{equation}
	\left|\lambda_j(G(\tau,k))\right| \leq 1+M\tau,\quad j=1,2,\cdots,p,
\end{equation}
其中$\left|\lambda_j(G(\tau,k))\right|$表示$G(\tau,k)$的特征值,$M$为常数。
\end{Theorem}\par
因此,要$|G(\tau,k)|\leq 1$,即
\begin{equation*}
 -1\leq\dfrac{1-4(1-\theta)a\lambda\sin^2\dfrac{kh}{2}}{1+4\theta a\lambda\sin^2\dfrac{kh}{2}}\leq 1
\end{equation*}
考虑左边的不等式,得
\begin{equation*}
4a\lambda(1-2\theta)\sin^2\dfrac{kh}{2}\leq2
\end{equation*}
因为$\sin^2\dfrac{kh}{2}\leq 1$,要求化为
\begin{equation}
 2a\lambda(1-2\theta)\leq 1.
\end{equation}
这是~\ref{eq:04_cxsks_js}~的稳定性要求.
\subsection{三层隐式格式}
考虑三层隐式差分格式,考虑
\begin{equation}\label{eq:04_cxsks_scys}
\dfrac{3}{2}\dfrac{u_j^{n+1}-u_j^n}{\tau}-\dfrac{1}{2}\dfrac{u_j^n-u_j^{n-1}}{\tau}
-a\dfrac{u_{j+1}^{n+1}-2u_{j}^{n+1}+u_{j-1}^{n+1}}{h^2}=0
\end{equation}
改写成便于计算的格式
\begin{equation}
 (3+4a\lambda)u_j^{n+1}-2a\lambda(u_{j+1}^{n+1}+u_{j-1}^{n+1})=4u_j^n-4u_j^{j-1}
\end{equation}
其中,~$\lambda=\tau/h^2$,易证差分格式~\ref{eq:04_cxsks_scys}~以二阶的精度逼近原微分方程.\par
将其化为等价的二层差分方程组
\begin{equation*}
\begin{cases}
(3+4a\lambda)u_j^{n+1}-2a\lambda(u_{j+1}^{n+1}+u_{j-1}^{n+1})=4u_j^n-v_j^n \\
v_j^{n+1}=u_j^n
\end{cases}
\end{equation*}
求得其增长矩阵为
\begin{equation*}
 G(\tau,k)=\begin{bmatrix}
  \dfrac{4}{3+8a\sin^2\dfrac{kh}{2}} & \dfrac{-1}{3+8a\sin^2\dfrac{kh}{2}} \\
                     1		     &			0		   \\
 \end{bmatrix}
\end{equation*}
其中,~$a=a\lambda$,$G$的特征方程为
\begin{equation*}
 \mu^2-\dfrac{4}{3+8\alpha\sin^2\dfrac{kh}{2}}\mu+\dfrac{1}{3+8\alpha\sin^2\dfrac{kh}{2}}=0
\end{equation*}
得$|\mu_i|\leq1(i=1,2)$.写出$\mu_i$的表达式
\begin{equation*}
\mu_{1,2}=\dfrac{2\pm\sqrt{1-8\alpha\sin^2\dfrac{kh}{2}}}{3+8\alpha\sin^2\dfrac{kh}{2}}.
\end{equation*}
若解为重根,显然有$|\mu_i|$,由此得出差分格式~\ref{eq:04_cxsks_scys}~是无条件稳定的.
\subsection{初边值问题}
我们考虑第一类边值问题,有
\begin{equation}
\begin{cases}
\dfrac{\partial u}{\partial t}-a\dfrac{\partial^2 u}{\partial x^2}=0,\quad 0<x<1,t>0 \\
u(x,0)=g(x),\quad 0\leq x\leq 1 \\
u(0,t)=\phi(t),\quad t\geq 0 \\
u(1,t)=\psi(t),\quad t\geq 0 
\end{cases}
\end{equation}
其计算区域为$x=[0,1]$,因此我们分开区间$0=x_0<x_1<\cdots<x_J=1$,第一边值问题的边界处理可取
\begin{equation}
\begin{cases}
u_0^n=\phi(t_n),\quad n\geq0 \\
u_J^n=\psi(t_n),\quad n\geq0
\end{cases}
\end{equation}\par
在初始线我们利用初始条件的离散
\begin{equation}
 u_j^0=g(x_j)=g_j
\end{equation}
得到边界点上的差分格式.\par
由于本研究不涉及第三类贬值条件,这一方面的差分处理方法就不叙述了.

\section{对流扩散方程}
考虑一个简单的对流扩散方程
\begin{equation}\label{equ:cf_dkfangc}
	\dfrac{\partial u}{\partial t}+a\dfrac{\partial u}{\partial x}=\nu\dfrac{\partial^2 u}{\partial x^2}
\end{equation}
其中a、$\nu$ 为常数,$\nu>0$,给定初值
\begin{equation}
	u(x,0)=g(x)
\end{equation}
构成了\textbf{对流扩散方程}的初值问题,在我们的模型中,它是一个对流占优的扩散问题。\par
\subsection{中心显式差分格式}
将方程~\ref{equ:cf_dkfangc}~差分,有
\begin{equation}\label{equ:cf_dkfangct}
	\dfrac{u^{n+1}_j-u^{n}_{j}}{\tau}+a\dfrac{u^{n}_{j+1}-u^n_{j-1}}{2h}=\nu\dfrac{u^n_{j+1}-2u^n_j+u^n_{j-1}}{h^2}
\end{equation}
其截断误差为$O(\tau+h^2)$。\par
下面来分析差分格式~\ref{equ:cf_dkfangct}~的稳定性,令
\begin{equation}
	\lambda = a\dfrac{\tau}{h},\quad\mu=\nu\dfrac{\tau}{h^2}
\end{equation}
则差分格式~\ref{equ:cf_dkfangct}~改写为
\begin{equation}\label{equ:cf_dkmmm}
u^{n+1}_j=u^n_j-\frac{1}{2}\lambda(u^n_{j+1}-u^n_{j-1})+\mu(u^n_{j+1}-2u^n_j+u^n_{j-1})
\end{equation}\par
式~\ref{equ:cf_dkmmm}~的增长因子为
\begin{equation}
	G(\tau,k)=1-2\mu(1-\cos kh)-i\lambda\sin kh
\end{equation}
模的平方为
\begin{equation}
\begin{aligned}
	|G(\tau,k)|^2 &= [1-2\mu(1-\cos kh)]^2+\lambda^2\sin^2 kh\\
				  &= 1-4\mu(1-\cos kh)+4\mu^2(1-\cos kh)^2+\lambda^2\sin^2 kh\\
				  &= 1-(1-\cos kh)[4\mu-4\mu^2(1-\cos kh)-\lambda^2(1+\cos kh)]
\end{aligned}				  
\end{equation}
差分格式稳定的充分条件为
\begin{equation}
4\mu-4\mu^2(1-\cos kh)-\lambda^2(1+\cos kh) \geq 0
\end{equation}
即
\begin{equation}
(2\lambda^2-8\mu^2)\dfrac{1-\cos kh}{2}+4\mu-2\lambda^2 \geq 0
\end{equation}
注意到$\dfrac{1}{2}(1-\cos kh)\in[0,1]$,上式应满足
\begin{equation}
(2\lambda^2-8\mu^2)+4\mu-2\lambda^2 \geq 0,\quad 4\mu-2\lambda^2 \geq 0
\end{equation}
由此得到差分格式~\ref{equ:cf_dkfangct}~的稳定性限制为
\begin{equation}
\tau \leq \dfrac{2\nu}{a^2}
\end{equation}
\begin{equation}
\nu\dfrac{\tau}{h^2}=\dfrac{1}{2}
\end{equation}\par
\subsection{迎风差分格式}
根据中心显式差分格式的稳定性要求可知,当$\nu/a^2$比较小时,时间步长是比较小的.我们在一阶空间偏导数的离散中采用单边差商,
令$a>0$,得到方程~\ref{equ:cf_dkfangc}~的迎风差分格式为
\begin{equation}\label{eq:cf_dkfangc_yf}
 \dfrac{u_j^{n+1}-u_j^n}{\tau}+a\dfrac{u_j^n-u_{j-1}^n}{h}=\mu\dfrac{u_{j+1}^n-2u_j^n+u_{j-1}^{n}}{h^2}
\end{equation}
容易看出,其截断误差为$O(\tau+h)$.\par
讨论其稳定性,将上式变为
\begin{equation}
 \dfrac{u_j^{n+1}-u_j^n}{\tau}+a\dfrac{u_{j+1}^n-u_{j-1}^n}{2h} = \left(\mu+\dfrac{ah}{2}\right)
 \dfrac{u_{j+1}^{n}-2u_j^n+u_{j-1}^n}{h^2}
\end{equation}
令$\bar{\mu}=\mu+\dfrac{ah}{2}$,可以得到式~\ref{eq:cf_dkfangc_yf}~的稳定性条件为
\begin{equation*}
\begin{split}
  \tau\leq&\dfrac{2}{a^2}\bar{\mu} \\[0.7em]
  \bar{\mu}\dfrac{\tau}{h^2}\leq&\dfrac{1}{2}
\end{split}
\end{equation*}
注意到,第一个稳定性条件等价于$\dfrac{\tau}{\dfrac{2\mu}{a^2}+\dfrac{h}{a}}\leq 1$,而
\begin{equation*}
\dfrac{\tau}{\dfrac{2\mu}{a^2}+\dfrac{h}{a}}=\dfrac{2\mu\dfrac{\tau}{h^2}\left(\dfrac{ah}{2\mu}\right)^2}{1+\dfrac{ah}{2\mu}}
\end{equation*}
利用不等式
\begin{equation*}
\dfrac{\left(\dfrac{ah}{2\mu}\right)^2}{1+\dfrac{ah}{2\mu}}\leq 1+\dfrac{ah}{2\mu}
\end{equation*}
这样得到
\begin{equation*}
\dfrac{\tau}{\dfrac{2\mu}{a^2}+\dfrac{h}{a}}\leq 2\mu\dfrac{\tau}{h^2}\left(1+\dfrac{ah}{2\mu}\right)
\end{equation*}
因此差分格式的稳定性条件为
\begin{equation}
 \left(\mu+\dfrac{ah}{2}\right)\dfrac{\tau}{h^2}\leq\dfrac{1}{2}
\end{equation}\par
迎风差分格式主要用于对流占优的扩散问题,采用中心差分格式不能很好地得出结果,采用迎风格式可以计算出问题的近似解.
\subsection{隐式差分格式}
考虑到隐性差分格式带来的诸多好处(如无条件稳定),给出迎风差分格式的隐式格式为
\begin{equation}\label{eq:04_yfcf_ys}
 \dfrac{u_j^{n+1}-u_j^n}{\tau}+a\dfrac{u_{j+1}^{n+1}-u_{j-1}^{n+1}}{2h}=\nu\sigma\dfrac{u_{j+1}^{n+1}
 -2u_j^{n+1}+u_{j-1}^{n+1}}{h^2}
\end{equation}
此格式是无条件稳定的.\par
现在考虑隐式差分格式的解法,将式~\ref{eq:04_yfcf_ys}~化为
\begin{equation*}
-\left(\dfrac{a}{2h}+\dfrac{\nu\sigma}{h^2}\right)u_{j-1}^{n+1}+\left(\dfrac{a}{2h}-\dfrac{\nu\sigma}{h^2}\right)u_{j+1}^{n+1}
+\left(\dfrac{1}{2}+\dfrac{2\sigma}{h^2}\right)u_j^{n+1}=\dfrac{1}{2}u_j^n
\end{equation*}
令
\begin{equation*}
\begin{aligned}
A_j &= -\dfrac{a}{2h}-\dfrac{\nu\sigma}{h^2} & \qquad & B_j = \dfrac{1}{2}+\dfrac{2\sigma}{h^2} \\[0.6em]
C_j &= \dfrac{a}{2h}-\dfrac{\nu\sigma}{h^2}  &  \qquad & D_j = \dfrac{1}{2}   \\
\end{aligned}
\end{equation*}
则方程可以写为
\begin{equation}\label{eq:04_yfcf_ys_b}
 A_j u_{j-1}^{n+1} + B_j u_j^{n+1} + C_j u_{j+1}^{n+1} = D_j
\end{equation}
其中,$j=1,2,\ldots,J-1$.我们令
\begin{equation}\label{eq:04_yfcf_ys_bj}
 u_0^{n+1}=0,\quad u_J^{n+1}=0
\end{equation}
为边界条件,方程~\ref{eq:04_yfcf_ys_b}~和方程~\ref{eq:04_yfcf_ys_bj}~构成系数为三对角矩阵的方程组,我们可以利用
追赶法求解这一方程组.\par
我们给出三对角矩阵存在Crout分解的充分条件,有
\begin{Theorem}[Crout分解的充分条件]
对于三对角矩阵
\begin{equation*}
A=\begin{bmatrix}
   b_1 & c_1 \\
   a_2 & b_2 & c_2 \\
       & a_3 & b_3    & c_3 \\
       &     & \ddots & \ddots  & \ddots            \\
       &     &        & a_{n-1} & b_{n-1} & c_{n-1} \\
       &     &        &         &   a_n   & b_n     \\
  \end{bmatrix}
\end{equation*}
存在Crout分解的充分条件为$a_i\not=0,(i=2,3,\ldots,n),c_1\not=0$,且
\begin{equation*}
\begin{split}
|b_1|&>|c_1|>0 \\
|b_i|&>|a_1|+|c_i|,(i=2,3,\ldots,n-1) \\
|b_n|&>|a_n|   \\
\end{split}
\end{equation*}
\end{Theorem}\par
方程~\ref{eq:04_yfcf_ys_b}~和方程~\ref{eq:04_yfcf_ys_bj}~的追赶法
求解充分条件为
\begin{equation}
\begin{cases}
|B_1|>|C_1|>0,\\
|B_j|\geq |A_j|+|C_j|,\quad A_j C_j\not=0,\quad j=2,\ldots,J-2,\\
|B_{J-1}|>|A_{J-1}| \\
\end{cases}
\end{equation}\par
利用矩阵的Crout分解解方程组,差分后得到的方程组可以写为
\begin{equation}~\label{eq:04_yfcf_jgmt}
\begin{bmatrix}
   B_1 & C_1 \\
   A_2 & B_2 & C_2 \\
       &  & \ddots    &  \\
       &     & A_j &  B_j    & C_j          \\
       &     &        &  & \ddots     \\
       &     &        &        &   A_{J-1}   & B_{J-1}     \\
\end{bmatrix}
\begin{bmatrix}
u_0^{n+1} \\
u_1^{n+1} \\
\vdots    \\
\vdots    \\
u_{J-2}^{n+1} \\
u_{J-1}^{n+1} \\
\end{bmatrix}
=
\begin{bmatrix}
D_0 \\
D_1\\
\vdots    \\
\vdots    \\
D_{J-2} \\
D_{J-1} \\
\end{bmatrix}
\end{equation}
对式~\ref{eq:04_yfcf_jgmt}~中的矩阵系数$A$进行Court分解,得
\begin{multline}
\begin{bmatrix}
   B_1 & C_1 \\
   A_2 & B_2 & C_2 \\
       &  & \ddots    &  \\
       &     & A_j &  B_j    & C_j          \\
       &     &        &  & \ddots     \\
       &     &        &        &   A_{J-1}   & B_{J-1}     \\
\end{bmatrix}=\\
\begin{bmatrix}
l_1 & \\
m_2 & l_2 \\
    & m_3 & l_3 \\
    &     & \ddots & \ddots & \\
    &     &        & m_{J-2} & l_{J-1} \\
    &	  &	   &	     &  m_{J-1} & l_{J-1} \\
\end{bmatrix}
\begin{bmatrix}
1 & u_0       \\
  &  1  & u_1 \\
  &     &   1  &  u_2 \\
  &     &      &   \ddots & \ddots \\
  &     &      &          &    1   & u_{J-1} \\
  &     &      &          &        &      1   \\
\end{bmatrix}
\end{multline}
其中,
\begin{equation}
\begin{cases}
m_i=a_i,\quad (i=2,3,\ldots,J-1) \\
l_1=b_1,u=c_1/l_1 \\
l_i=b_i-m_iu_{i-1},\quad (i=2,3,\ldots,J-1) \\
u_i=c_i/l_i,\quad (i=2,3,\ldots,J-2) 
\end{cases}
\end{equation}\par
利用
\begin{equation}
\begin{bmatrix}
l_1 & \\
m_2 & l_2 \\
    & m_3 & l_3 \\
    &     & \ddots & \ddots & \\
    &     &        & m_{J-2} & l_{J-1} \\
    &	  &	   &	     &  m_{J-1} & l_{J-1} \\
\end{bmatrix}
\begin{bmatrix}
y_0 \\
y_1 \\
y_2 \\
\vdots \\
y_{J-2} \\
y_{J-1} \\
\end{bmatrix}=
\begin{bmatrix}
 d_0 \\
 d_1 \\
 d_2 \\
 \vdots \\
 d_{J-2} \\
 d_{J-1} \\
\end{bmatrix}
\end{equation}
其中,
\begin{equation}
\begin{cases}
y_1=d_1/l_1 \\
y_i=(d_i-m_iy_{i-1})/l_i,\quad (i=2,3,\ldots,J-1)
\end{cases}
\end{equation}
解得$Y^{\mathrm{T}}$.\par
将$Y^{\mathrm{T}}$代入
\begin{equation}
\begin{bmatrix}
1 & u_0       \\
  &  1  & u_1 \\
  &     &   1  &  u_2 \\
  &     &      &   \ddots & \ddots \\
  &     &      &          &    1   & u_{J-1} \\
  &     &      &          &        &      1   \\
\end{bmatrix}
\begin{bmatrix}
x_0 \\
x_1 \\
x_2 \\
\vdots \\
x_{J-2} \\
x_{J-1} \\
\end{bmatrix}=
\begin{bmatrix}
y_0 \\
y_1 \\
y_2 \\
\vdots \\
y_{J-2} \\
y_{J-1} \\ 
\end{bmatrix}
\end{equation}
其中,
\begin{equation}
\begin{cases}
x_n=y_n \\
x_i=y_i-u_ix_{i+1},\quad (i=J-1,J-2,\ldots,1)
\end{cases}
\end{equation}
就解得了$X^{\mathrm{T}}$.
\section{多维度方程}
现在我们开始讨论在空间上建立和求解数学模型,根据运动控制方程,我们得到物质在空间上的分布为
\begin{equation}
 \dfrac{\partial u}{\partial t}=a\left(\dfrac{\partial^2 u}{\partial x^2}+\dfrac{\partial^2 u}{\partial y^2}
 +\dfrac{\partial^2 u}{\partial z^2}\right)-b\left(\dfrac{\partial u}{\partial x}+\dfrac{\partial u}{\partial y}
 +\dfrac{\partial u}{\partial z}\right)+cu+d
\end{equation}
