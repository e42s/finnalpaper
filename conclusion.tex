\chapter{结论与展望}
在本文中,首先我们通过分析对微生物迁移过程影响因素的分析得到了微生物在土壤中运移过程的二阶偏微分方程模型,
即对流扩散反应模型.通过实验测定和文献调研得到了模型中的各项参数,我们得到了该模型具有对流占优的特点.\par
然后,我们重点研究了如何求解对流占优的对流扩散反应模型.逐步求得了扩散方程、对流方程和非齐次方程的解析解,为
之后研究数值求解方法提供了对照.研究了扩散方程、对流方程和对流扩散方程的有限差分解法,讨论了各类常见的
差分格式的应用,并证明了它们的稳定性条件和精确度.综合上面的结论,我们采用算子分裂法将三维对流扩散反应模型
分解为扩散方程、对流方程和反应方程,分别对它们求解后进行叠加,得到了最终的三维对流扩散反应模型的求解算法.\par
我们还利用MATLAB等工具对上面的求解方法进行数值模拟与验证,并在能够求得解析解的算例(如扩散方程、对流方程
和对流扩散方程)上进行了解析解和数值解的比对.实践证明我们采用的差分格式及相应的算法是简单有效的.\par
最后,
另外,我们对差分方程的计算法进行了简要的介绍,也介绍了计算过程中所遇到的基本理论.对于其中重要的定理与
结论,我们也进行了定理的证明与推导.\par
虽然我们在本文中讨论的是基于偏微分方程的运动模型,采用的机理建模法.模型虽然考虑了诸多因素在内,但是仍然
忽略了许多因素.如果能够采用系统辨识、人工神经元网络、支持向量机等黑箱方法作为补充,相信能够更好地描述微生物的
运动情况,使模型更加准确,这样才能得到更好的仿真分析的结果.\par
对流扩散反应方程不仅仅能够描述物质运移的过程,描述粘性物质流动的Navier-Stokes方程和热弹性介质波动过程的
Burgers方程也属于对流扩散反应方程类型的.所以,研究对流扩散反应方程的求解方法是十分有必要的.
本文的工作仅仅讨论了部分求解方法,对于更加现代的解法并没有加以讨论与研究.相信采用现代数学的方法对本问题进行研究,
能够得到精确度更高、稳定性更好,更加经济与快速的解法.