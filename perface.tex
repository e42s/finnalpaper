\chapter*{前\qquad 言}
土壤微生物是自然物质循环不可缺少的成员,在有机物质的矿物化、腐殖质的形成和分解、
植物营养元素的转化等诸多过程中起着不可替代的作用.微生物物质虽然仅占土壤有机组成的一小部分,
但它是生活着的有机体和物质转化者,也是土壤肥力的重要因素,对植物养料具有储存和调节作用.
土壤微生物主要指土壤中那些个体微小的生物体,主要包括细菌、放线菌、真菌还有酵母菌等,
它们在土壤形成和演化过程中起着主导作用,并且不同类型的土壤中的微生物种类和数量也相应不同,
由于土壤环境条件复杂等原因,目前为止仅1\%--10\%的土壤微生物被分离和鉴定,
这限制了对土壤微生物在陆地生态系统中重要作用的认识.\par
近年来,国内外学者在土壤微生物季节动态、不同类型土壤微生物数量、区系以及种群动态变化等方面作了较系统的研究,
20世纪40年代初已开始对土壤中微生物吸附和运移进行研究,早起的研究工作主要集中在传染病、垃圾处理、污水灌溉处理、
用渗滤方法纯化受微生物和有机物污染的水等方面.现已扩展到原位生物修复(将分离的到的或基因工程合成的细菌或营养物质注入土壤
,使其运移到受污染处并大量繁殖,分解有毒物质,达到治理目的)、提高石油开采量、放射性物质和有机污染物的携带运移、
提高冶炼率(用菌液可增加某些矿物的淋溶程度,提高溶液中金属含量)、核废料处理和根系层内病害的生物防治及养分的转化等方面
.\par
微生物在土壤中的运移过程看似简单,实际很复杂.其运移机理包括生长、吸附、解吸、沉积(过滤、布朗扩散、截流、沉降)、
腐解、钝化、滞留等过程,确定微生物运移速率、时间、分布范围,最大限度地提高细菌降解作用,减少细菌本身的再污染,
有必要对微生物在土壤中运移及其影响因素进行定量研究,建立数学模型.\par
通过对数学模型的求解和分析,得到微生物运动情况的动态分析结果,能够较好地预测微生物的运移情况,分析
影响微生物运移过程的因素,从而对实验和生产提出参考意见,为相关实验研究提供了帮助.\par
在本文中,我们首先就微生物运移情况进行了简要的介绍,定性地分析了各项因素对微生物运动的影响情况,介绍了在研究中
涉及的数学理论和使用的数学工具.然后,根据质量守恒定律建立了微生物在土壤中运移过程的偏微分方程描述的数学模型.
之后,我们讨论了求解精确解和数值解的方法和步骤,证明了一系列结论,推导了求解差分格式.最后,利用计算机进行数值
模拟,初步地得到了部分微生物在土壤中运移情况的图像,为今后的研究打下了基础.
